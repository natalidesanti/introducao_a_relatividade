\documentclass[11pt,a4paper,titlepage,brazil]{article}
\usepackage[brazilian]{babel}
\usepackage[T1]{fontenc}
\usepackage[utf8]{inputenc}
\usepackage[a4paper]{geometry}
\geometry{top=2.0cm, bottom=2.0cm, left=2.0cm, right=2.0cm}
\usepackage{setspace}
\usepackage{libertine}

\begin{document}

\begin{center}
 \large{\textbf{Resumo minicurso: Introdução à relatividade}}\\
 Natalí Soler Matubaro de Santi
\end{center}

Neste minicurso abordarei os princípios básicos da teoria da relatividade, incluindo a notação covariante e a terminologia da dinâmica relativísitca. Em relatividade especial abordarei tópicos desde as transformações de Lorentz, a contração espacial e a dilatação temporal até a conservação de energia e momentum. Em relatividade geral introduzirei o conceito de espaço-tempo curvo, o tensor de Riemann, Ricci e seu escalar, as esquações de Einstein e, por fim, a solução de Schwarzschild.

\end{document}
%%% Local Variables:
%%% mode: latex
%%% TeX-master: t
%%% End:
