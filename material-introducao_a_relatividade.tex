\documentclass[12pt,a4paper,titlepage,brazil]{article}
\usepackage{tabularx}
\usepackage{graphicx}
\usepackage[brazilian]{babel} 
\usepackage[T1]{fontenc}
\usepackage[utf8]{inputenc}
\usepackage[a4paper]{geometry}
\geometry{top=2.5cm, bottom=2.5cm, left=3.0cm, right=2.0cm}
\usepackage{setspace}
\usepackage{titleps}
\usepackage{amsmath}
\usepackage{libertine}
\usepackage{tcolorbox}%para fazer caixas no texto - Gera caixas em escalas de cinza
%\newtcolorbox{mybox}{colback=red!5!white,colframe=red!75!black}%gera caixas com as cores definidas
\newtcolorbox{mybox1}[1]{colback=red!5!white,colframe=red!75!black,fonttitle=\bfseries,title=#1}%um tipo customizado de caixa com titulo e cores
\newtcolorbox{mybox2}[1]{colback=green!5!white,colframe=green!75!black,fonttitle=\bfseries,title=#1}%um tipo customizado de caixa com titulo e cores

%Cabeçalho e o rodapé
\newpagestyle{ruled}
{\sethead{}{Minicurso: Introdução à relatividade}{}\headrule
  \setfoot{}{Natalí Soler Matubaro de Santi --- natalidesanti@gmail.com}{}\footrule}
\pagestyle{ruled}

%-----------------------------------------------------------------------

\begin{document}

\begin{center}
 \LARGE{\bf Introdução à relatividade}
\end{center}

%------------------------------------------------------------------------

\section{Relatividade especial}

%------------------------------------------------------------------------

\subsection{Transformações de Lorentz}

%------------------------------------------------------------------------

\subsection{Contração espacial}

%------------------------------------------------------------------------

\subsection{Dilatação temporal}

%------------------------------------------------------------------------

\subsection{Dinâmica relativística: convervação de energia e momentum}

%------------------------------------------------------------------------

\subsection{Notação covariante}

%------------------------------------------------------------------------

\section{Relatividade geral}

%------------------------------------------------------------------------

\subsection{Espaço-tempo e a notação tensorial}

%------------------------------------------------------------------------

\subsection{Tensor de Riemann}

%------------------------------------------------------------------------

\subsection{Tensor de Ricci}

%------------------------------------------------------------------------

\subsection{Escalar de Ricci}

%------------------------------------------------------------------------

\subsection{Equações de Einstein}

%------------------------------------------------------------------------

\subsection{Solução de Schwarzshild}

%------------------------------------------------------------------------

%------------------------------------------------------------------------

\end{document}
%%% Local Variables:
%%% mode: latex
%%% TeX-master: t
%%% End:
